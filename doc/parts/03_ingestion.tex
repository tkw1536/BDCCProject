\label{sec:ingestion}

\subsection{The NIfTI format}

After viewing the image viewer it turned out that the HDF / BBIC encoded data is not the original data the brain scans were generated from. It was generated from files in NIfTI format. 

NIfTI \cite{NIfTI:website}, or Neuroimaging Informatics Technology Initiative, is a format created by a group of the same name. It is very commonly used for Neuroimages. As suggested by the name it is commonly used as the format of neuro images. Each file contains a single multi-dimensional array of homogeneous type as well as domain specific meta-data. 

This format is a huge improvement on the BBIC format because it gives us direct access to the data we want to ingest into Rasdaman. In our case this gave us direct access to a three-dimensional array of floats. This allowed us to ingest the data in a collection of type \lstinline{FloatSet3} \footnote{Rasdaman organises the multi-dimensional arrays it stores in so-called collections. Each collection corresponds to a list of multi-dimensional arrays of the same type. The type \lstinline{FloatSet3} corresponds to a collection of three-dimensional arrays with floating point values. }. 

Rasdaman does not support this format natively. Hence in order to ingest the data we need to find a way to access the data format and then convert it into a format that Rasdaman can understand. 

\subsection{Reading Data from Python}

In order to access the data we used Python. A convenient library for acessing neuroimaging data called Nibabel is available from \cite{Nibabel:website}. It provides interoperatbility with the Numpy / SciPy stack and is thus perfeclty suited for our purposes. This allowed us to read in and access an image file via the following code: 
\begin{lstlisting}
>>> import nibabel as nb # Load the NiBabel Library
>>> import numpy as np # Load the Numpy Library
>>>
>>> image = nb.load("colin27_T1_seg.nii") # load the image file
>>>
>>> data.dtype # print the datatype of the image
dtype('float64')
>>> data.shape # print the data shape
(151, 188, 154)
\end{lstlisting}

\subsection{Suitable Data formats for Rasdaman}

In terms of formats that Rasdaman supports that can be used to read in the data, two different formats come to mind, CSV and TIFF. 

CSV, or comma seperated values, is a format that encodes up to two-dimensional data inside an ASCII file. For this it turns the data values into strings and joins them with commas. In the case of two-dimensional data, each line represents a single row of the dataset. This is a very wasteful encoding and unfortunatly not well suited for our data because it is three-dimensional. While Rasdaman supports three-dimensional CSV, it is non-standard and would thus proved to be too difficult to use. 

TIFF, or Tagged Image File Format, is a lossless image raster format that was originally created by Aldus Corporation and is now maintained by Adobe \cite{tiff:website}. TIFF in theory supports both two and three-dimensional images however Rasdaman only supports the former. Furthermore it is possible to write to TIFF files from python using a library called \lstinline{tifffile.py} available at \cite{tifffile:website}. 

\subsection{From Python to Rasdaman}

In order to work around the fact that Rasdaman supports only two-dimensional TIFF files, we decided to import the data using so-called partial updates. This involves slicing the three-dimensional array into two-dimensional TIFF files which can then be imported on by one into Rasdaman. Using the \textit{rasql} tool to insert data into Rasdaman this involves three steps. 

\begin{lstlisting}[showstringspaces=false,morekeywords={NAME},language=Bash]
# Step 1: Create a collection of type FloatSet3 inside Rasdaman
rasql -q "create collection NAME FloatSet3"

# Step 2: Insert an (empty) multi-dimensional array into the collection
rasql -q "insert into NAME values marray it in [0:0,0:0,0:0] values 0f"

# Step 3: Read each TIFF slice using partial updates
rasql -q "update NAME as c set c[INDEX,*:*,*:*] assign inv_tiff(\$1)" --file FILENAME.tiff
\end{lstlisting}

In the first step we create a collection of type \lstinline{FloatSet3} inside Rasdaman. For this we need to give it a name (we just used a placeholder) and the collection type. 
In the second step we create an empty two-dimensional array inside the collection. 
In the third step (which we need to repeat for each slice) we update a single slice of the data cube with data read from a tiff file. 

\subsection{Automating Ingest}

Since we usually have several hundered slices we do not want to do this manually. Thus we have written a python script that combines this process into a single step. The script consists of about 200 lines of python code and is well documented. It is called \lstinline{nii_to_tiff.py} and concretly performs the following steps: 

\begin{enumerate}
  \item Reads a NIfTI file, 
  \item splits it into different slices and store each one as a TIFF file, 
  \item output a bash script to STDOUT which calls \lstinline{rasql} with the arguments mentioned above to import the slices and 
  \item delete the TIFF files since they are no longer needed. 
\end{enumerate}

In order to import a single file we can then use the command
\begin{lstlisting}[showstringspaces=false,morekeywords={NAME},language=Bash]
python3 nii_to_tiff.py FILENAME . | rasql_coll=COLLECTION rasql_args="--quiet" bash
\end{lstlisting}