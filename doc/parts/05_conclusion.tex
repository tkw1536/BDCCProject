\label{sec:outlook_conclusion}

\ednote{Write this}

% \begin{figure}[h]
% \centering
% \begin{subfigure}{.4\textwidth}
%   \centering
%   \includegraphics[width=\textwidth]{imgs/cepheid}
%   \caption{Cepheid Stars are known to flash occasionally. Such a flash is an example of an oscilation. }
% \end{subfigure}%
% \space\space\space
% \begin{subfigure}{.5\textwidth}
%   \centering
%   \includegraphics[width=\textwidth]{imgs/oscillation}
%   \caption{An pendulum swinging to the left and right and a spring oscilating up and down. }
% \end{subfigure}
% \caption{Some examples of oscilating objects}
% \label{fig:intro_samples}
% \end{figure}

% \begin{lstlisting}
% # create a new graph
% G = nx.DiGraph()
% 
% # read current parameters
% A = self.A
% N = self.N
% 
% # add all the edges
% for i in range(N):
% 		for j in range(N):
% 				if A[i, j] != 0:
% 						G.add_edge(i, j, weight=A[i,j])
% 
% # and return the graph
% return G
% \end{lstlisting}